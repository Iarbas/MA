\chapter{Zusammenfassung und Ausblick}
In Zukunft werden Indoor-Lokalisierungssysteme eine immer wichtigere Rolle einnehmen. Im Hinblick auf die Vernetzung der Gesellschaft gehört sie in die heutige Zeit. Derzeit sind jedoch noch viele Herausforderungen zu meistern, speziell in Anbetracht einer effektiven Planung und eines ökonomischen Umgangs mit dieser Technologie. Dafür müssen mehrere Schwerpunkte verfolgt werden:
\begin{itemize}
\item Erstellung geeigneter Werkzeuge für Experimente
\item Standardisierbare Verfahren für Messungen
\item Definition von Gütekriterien für Lokalisierungssysteme
\end{itemize}
Die vorliegende Arbeit befasst sich mit genau dieser Problematik. Das Ziel ist es, die Infrastruktur für ein Lokalisierungssystem anhand von fest definierten Kriterien optimal auszulegen. Zudem wird die Möglichkeit und Methodik einer vollautomatisierten Validierung eines Indoor-Ortungssystems beschrieben.\\ \\
Bisherige Ansätze betrachten immer nur einen Teil der Herausforderungen, aber es fehlt ein komplettes Konzept dafür, das jeden Aspekt berücksichtigt und beherrscht. Deshalb erforderte es, ein eigenes Konzept zu entwickeln und einen tieferen Einblick in die Thematik zu wagen. Dieses Vorhaben wird Lighthouse Keeper genannt und umfasst die Planung von benötigter Hardware und Software. Um die Daten so nah wie möglich am Endverbraucher auszuwerten, wurde ein Smartphone als Empfangsgerät genutzt und eigens dafür eine Applikation geschrieben. Als Helfer für die Versuche wurden zwei Roboter, Youbot und Scitos G5 akquiriert, durch den der Prozess standardisierbar wurde. Die Kommunikation zwischen allen Plattformen wird mithilfe des Frameworks ROS verwirklicht. Mit der nötigen Hard- und Software konnten die ersten Messwerte aufgenommen werden und aus ihnen ein Modell des Ausbreitungsverlustes von BLE-Signalen erschlossen werden. Mit diesem Modell konnten aus den empfangenen RSSI-Werten die Distanz zwischen Sende- und Empfangsgerät geschätzt werden. Die softwaretechnische Auswertung fand dabei stets offline am PC statt. Deswegen lag der Fokus zunächst auf statischen Entfernungen. Die Langzeituntersuchungen von Beacon-Signalen häuften dabei Datensätze von mehr als 30 Stunden an, aus denen eine Vertrauensskala für die einzelnen Distanzen erstellt wurde. Die definierte Skala gibt an, wie lange an einer Distanz zu einem Beacon auf ein für das Modell gültiges Signal gewartet werden muss. Zudem wurden auch Untersuchungen für störende Faktoren eingeleitet, um Unsicherheitsfaktoren bei den Messungen zu minimieren. Im Anschluss wurde mithilfe der Vertrauensskala ein Optimalitätskriterium aufgestellt, anhand dessen die Beacons in einem Raum optimal verteilt werden. Der Prozess aus Experimenten, Modellierung und Simulation wird mit einer Evaluierung einer optimalen Beacon-Konfiguration gekrönt. Die Ergebnisse aus dem automatisierten dynamischen Experiment beweisen die Zuverlässigkeit des Modells und der Annahmen. Dadurch wurde die Funktionsfähigkeit und Korrektheit des Konzeptes bewiesen.\\ \\   
Die Ausführungen sind den Aufgaben in Bezug auf die Planung von Indoor-Infrastrukturen und der Durchführung von dynamischen Experimenten gerecht geworden. Es wurden alle Anforderungen bedient und es sind damit sehr gute Ergebnisse erreicht worden. Im Falle einer Weiterführung dieser Arbeit bieten sich folgende Schwerpunkte an:
\begin{itemize}
\item Validierung des Konzeptes unter Verändrungen der Beacon-Einstellungen (Sendeleistung und Sendefrequenz). Unter Betrachtung dieser Änderungen müssen ferner ökonomische Gutachten erstellt werden, die in die Optimierung mit einfließen.
\item  Erweiterung der Simulationsumgebung um Objekte im Raum. Zudem müssen die Befestigungen auch manuell gewählt werden können und die Beacon-Einstellungen zusätzlich auswählbar sein.
\item Verzicht auf proprietärer Software durch Entwicklung eigener Programme. Denn die Probleme mit der App und der mangelnde Zugriff auf viele Funktionen auf den verschiedenen Plattformen haben im Endeffekt die Arbeit erschwert. Der Effekt der Zeitersparnis durch die Verwendeung von fertiger Software, wurde nachträglich durch die verursachten Probleme der fremden Software wieder aufgezerrt. 
\item Erweiterung der Optimierung um Gewichtungen für die einzelnen Raumelemente. Dadurch kann der Fokus auf kritische Bereiche (z.B. Eingänge und Besonderheiten) gelegt werden. 
\item Weitere Untersuchungen des Ausbreitungsverlustes für größere Distanzen durchführen. Für größere Räume ist es ungewiss, ob das Modell noch dafür ausgelegt ist. Hierbei müssen die statischen Experimente erneut durchgeführt werden.
\end{itemize}
Der bisherige Mangel an einem umfassenden Planungskonzept wird hoffentlich mit dieser Arbeit zum Teil beseitigt. Es wäre wünschenswert, wenn diese Arbeit die Beacon-Technologie weiter voranbringt und andere Menschen für die weiterführende Forschung in diesem Bereich inspiriert.