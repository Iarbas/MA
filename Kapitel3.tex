\chapter{Konzeptplanung}
Nach der Betrachtung des Standes der Technik fällt auf, dass sowohl die Planung, als auch die Inbetriebnahme von den innerräumlichen Ortungssystemen stets ein experimentierlastiger Prozess ist. Es existieren weder bekannte Theorien, noch computergestützte Hilfsmittel für deren Auslegung und Validierung. Zum einen wird es daran liegen, dass es dafür noch keine Notwendigkeit gab und zum anderen, weil es aufgrund der Vielseitigkeit der Anwendungsgebiete und Situationen noch kein gemeinsamer Standard gefunden wurde. Dabei ist das Problem der Erstellung einer guten Indoor-Lokalisierung nicht trivial und hoch Komplex, wenn all die Faktoren hinzugerechnet werden, die das System beeinträchtigen können. Zudem verliert sich der Überblick über alle Signalquellen und deren Position in großen Gebäuden, weswegen auch noch keine Großprojekte diesbezüglich entstehen. Im folgenden soll in diesem Abschnitt ein Konzept entwickelt werden, welche die Theorie mit ergänzender Simulationstechnik im Zusammenspiel mit Experimenten unterstützt und somit ein bekanntes Schema in Wissenschaft und Industrie aufgreift: der Prozessplanung.
\section{Vorüberlegungen}
Der erste Schritt für eine systematische Prozessplanung wäre die Schaffung einer Theorie bzw. eines Modells für die Signalausbreitung der BLE-Signale. Dieses muss erst durch Betrachtung der physikalischen Eigenschaften und mathematischen Beziehungen geschaffen oder anhand eines existierenden Modells aus vergleichbaren Technologien abgeleitet werden. Nachdem ein Modell aufgestellt wurde, muss dieses parametrisiert werden, um das Modell auf das reale Verhalten besser abbilden zu können. Mithilfe des Modells lässt sich anschließend eine Simulation erstellen, die die Signalausbreitung von BLE-Signalen in einem Raum prädiktionieren und daraus die Beacons unter Erfüllung von Optimalitätsbedingungen anschließend automatisch verteilen kann. Die Anordnung der Beacons kann so je nach Einsatzszenario angepasst und vor der eigentlichen Installation simuliert und getestet werden. Jedoch muss die fertige Konfiguration erneut durch Messungen auf die gewünschten Funktionen und Qualität der Indoor-Lokalisierung hin geprüft werden. Dieser Kreislauf aus Experimenten, Modellierungen und Simulationen wird dazu in Abbildung \ref{fig:Prozessplanung} grafisch veranschaulicht. Der Vorteil dieser Herangehensweise ist die fundierte Lösung eines Problems, sodass auf die gewonnenen Informationen aufgebaut werden kann und sich mit ihnen wissenschaftlich arbeiten lässt. Für die Planung eines Indoor-Lokalisierungsproblems mittels Beacons bedeutet das, dass die Konfigurationen vorab geplant werden können und diese einheitlich aufgebaut sind. Dadurch ergeben sich auch kommerzielle Vorteile durch eine schnellere Entwicklung von Ortungs- und Navigations-Applikationen und weiterer Softwareprojekte. Zudem wird die Leistungsfähigkeit der Beacon-Technologie messbar, was für eine bessere Definition ihrer Einsatzmöglichkeiten führen wird.
\begin{figure}[H]
\centering
\begin{tikzpicture}[>=latex']
\def \n {3}
\def \radius {3cm}
\node[draw, rectangle, minimum height=1cm,minimum width=3cm, top color=red!40,
      bottom color=red!5, name path=t1, blur shadow={shadow blur steps=5}] at ({0+90}:\radius) {Experiment};
\node[draw, rectangle, minimum height=1cm,minimum width=3cm, top color=blue!40,
      bottom color=blue!5, name path=t2, blur shadow={shadow blur steps=5}] at ({360/\n+80}:\radius) {Simulation};
\node[draw, rectangle, minimum height=1cm,minimum width=3cm, top color=yellow!40,
      bottom color=yellow!5, name path=t3, blur shadow={shadow blur steps=5}] at ({360/\n * 2+100}:\radius) {Theorie};
\path[name path=k]circle[radius=\radius];
\path[name intersections={of=k and t1,sort by=k,by={i-1,i-2}}]
     [name intersections={of=k and t2,sort by=k,by={i-3,i-4}}]
     [name intersections={of=k and t3,sort by=k,by={i-5,i-6}}];
\draw[stealth-stealth,line width=8pt,black, postaction={draw,stealth-stealth,blue!50,line width=7pt, shorten <=1pt,shorten >=1pt}] (i-2) to[arc to=3cm] (i-3);
\draw[stealth-stealth,line width=8pt,black, postaction={draw,stealth-stealth,blue!50,line width=7pt, shorten <=1pt,shorten >=1pt}] (i-4) to[arc to=3cm] (i-5);
\draw[stealth-stealth,line width=8pt,black, postaction={draw,stealth-stealth,blue!50,line width=7pt, shorten <=1pt,shorten >=1pt}] (i-6) to[arc to=3cm] (i-1);
\end{tikzpicture}
\caption{Kreislauf einer Prozessplanung}
\label{fig:Prozessplanung}
\end{figure}
\section{Zielsetzungen}
Aus den Vorüberlegungen ergibt sich eine klar definierte Struktur, die nun konkret mit einzelnen Abläufen in diesem Absatz beschrieben wird. Die Zielsetzungen sind dabei zum einen, ein Modell für die Ausbreitung von BLE-Signalen zu finden und zu parametrisieren. Dafür braucht es ein systematisches Vorgehen für die Aufnahme der Messwerte und eine Plattform als Empfänger der BLE-Signale. Nachfolgend muss das Modell in eine Simulationsumgebung eingepflegt und ein Optimierungsalgorithmus für die Positionierung der Beacons implementiert werden. Im letzten Schritt soll als Abschluss der Arbeit ein Testfeld mit einer simulierten Konfiguration aufgebaut und die Messungen daraus mit den Simulationsergebnisse verglichen werden. Dafür wird ein mobiles System benötigt, welches vorgegebene Punkte im Feld des Lokalisierungssystems ansteuern und seine Position mit der errechneten Position aus der Trilateration vergleichen kann. Um alle diese Punkte zu erfüllen, müssen Systeme und Werkzeuge gefunden werden, die die Konzeptplanung unterstützen. Im Anschluss werden diese näher benannt und deren Zusammenspiel erläutert.
\subsection{Hardware-Anforderungen}
An das Empfängsgerät wird lediglich die Anforderung gestellt, dass die Messung eines Beacon-Signals mit der Messung einer Position synchronisierbar ist und somit über Netzwerkfähigkeit oder Speicher verfügen muss. Dies ist gerade in der Validierungsphase von hoher Wichtigkeit, um standardisierte und reproduzierbare Ergebnisse zu gewährleisten. Da die eigentliche Nutzung des Positionierungssystems mithilfe von Smartphones stattfindet, ist die Verwendung dieser Endgeräte für die Messungen durchaus zu empfehlen. Denn Smartphones besitzen eine große Bandbreite an Schnittstellen und einen internen Speicher, sodass jeweils eine Strategie für die Messwertaufnahme verfolgt werden kann. Zudem existieren von den Herstellern der Beacons bereit fertige Bibliotheken und Entwicklertools, um die Signale von ihren Beacons zu verarbeiten, was einen geringeren Aufwand für die Implementierung bedeutet. Zudem weisen Antennen der Empfangsgeräte Charakteristiken auf, die die Aufnahme der Messungen je nach Ausrichtung des Smartphones beeinflussen. Es gilt somit auch diese Einflüsse zu untersuchen und da die meisten Smartphones auch über Inertialsensoren verfügen, bieten sie gleich zusätzliche Sensordaten in einem Gerät. Die Problematik der Antennencharakteristiken findet sich dabei im nächsten Kapitel. Für die standardisierte Aufnahme der Messwerte bietet sich eine Kombination aus Maschine bzw. Roboter und Smartphone an. Da die Roboter-Systeme mobil sind, um eine Positionsbestimmung unabhängig von der Beacon-Ortung erweitert werden können und ebenfalls Kommunikationfähigkeiten durch entsprechende Middleware besitzen, bietet sich ihre Verwendung für die Überprüfung von Beacon-Konfigurationen an. Es ist dabei nur darauf zu achten, dass es eine entsprechende Halterung für das Smartphone am Roboter existiert. Darüber hinaus kann der gesamte Prozess durch die Verwendung von Robotern in den Experimenten automatisiert werden, welches ebenfalls ein Ziel dieser Arbeit darstellt.  
\subsection{Konzept der Software-Architektur}\label{sec:KonSofArch}
Im Mittelpunkt der Software-Architektur steht die Datenübertragung der Messwerte aus verteilten Systemen auf ein zentrales System, um diese später im Verarbeitungsschritt zu synchronisieren. Um dies zu gewährleisten müssen alle Plattformen miteinander kommunizieren können und die Art der Übertragung darf dabei nicht zu starken Verzögerungen der Messwertaufnahme führen. Sprich es wird ein effektives und schnelles Framework benötigt, dass die eigentliche Messungen nicht beeinträchtigt. Es muss dabei zwischen den Experimenten unterschieden werden, denn für die Parameterbestimmung des Modells wird keine ständige Positionsmessung benötigt, da diese "`per Hand"' ermittelt werden kann und sich im Laufe einer Messreihe für eine Distanz auch nicht ändert. Hingegen bei der Überprüfung eines bestehenden Lokalisierungssystem fließt die "`externe"' Lokalisierung eines Roboters in den Vektor aus Messwerten mit ein, sodass über das Framework der Kommunikation verschiedene Datentypen von den verschiedenen Plattformen auf die zentrale Verarbeitungsstelle geleitet werden müssen. Die Anforderungen an die einzelnen Anwendungen und unteren Ebenen der Software wird in Abbildung \ref{fig:SoftArch} einmal dargestellt. Die Anforderungen an die einzelnen Anwendungen sind bei dem Smartphone die Erfassung der einzelnen BLE-Signalen von den Beacons und der Messung der eigenen Ausrichtung. Im Gegensatz dazu muss der Roboter seine eigenen Positionsdaten übermitteln und zudem die auf einem externen Computer in der Simulation erstellten Raumelemente ansteuern und somit Befehle vom PC entgegennehmen. Die Simulationsumgebung speichert alle Sensordaten der verteilten Systeme und verarbeitet diese für die eigentliche Validierung der Beacon-Konfiguration. \\ \\
Da die Simulation zur Erstellung eines Indoor-Lokalisierungssystems in Gebäude bzw. Räumen angewendet wird, muss die Simulationumgebung zuerst den Grundriss eines Gebäudes analysieren und die gewonnenen Daten für die spätere Verwendung verarbeiten. Es bietet sich dabei an den Raum in einzelne Elemente zu unterteilen, sodass die Simulation vereinfacht wird und mit diskreten Werten gerechnet werden kann. Die Informationen über die Anordnung von Objekten und Wänden ist dabei essentiell für die manuelle oder automatische Verteilung der Beacons im Raum und der Simulation der einzelnen Signalstärken in Abhängigkeit zu den Distanzen der Raumelementen zu den Beacons. Der grundlegende Ablauf ist in Abbildung \ref{fig:SimArch} hierfür skizziert. Die Optimierung einer Anordnung von Beacons ist dabei stets subjektiv, da für jedes Szenario andere Anforderungen an das Ortungssystem gestellt werden. Es gilt somit ein Optimierungsalgorithmus zu finden, der mit unterschiedlichen Ansprüchen je nach Situation umgehen und gegebenenfalls auf ein spezielles Optimierungsproblem angepasst werden kann. Auf die Optimierung wird speziell in Kapitel 5 eingegangen.
\pgfdeclarelayer{background}
\pgfdeclarelayer{foreground}
\pgfsetlayers{background,main,foreground}
\tikzstyle{sensor}=[draw, fill=blue!20, text width=3cm, text centered, minimum height=2.5em]
\begin{figure}[t]
\centering
\begin{tikzpicture}
	\tikzset{
    	myarrow/.style={->, >=latex', shorten >=1pt, thick}, 
	}  
    \node[cloud, cloud puffs=15.7, text width=3cm, draw, fill=red!10, ] (cloud) at (0,0) {Kommunikations-Framework};
    \node[sensor] (BeaconSig) at (-6,1) {Beacon-Signale};
    \node[sensor] (IMU) at (-6,-1) {Inertial-sensoren};
    \node[above of=BeaconSig] (App) {Smartphone-App};
    
    \node[sensor] (Pos) at (0,-5) {Positionsdaten};
    \node[below of=Pos] (Robo) {Middleware vom Roboter};
    
    \node[sensor] (Verar) at (6,1) {Informations-Verarbeitung};
    \node[sensor] (Steuer) at (6,-1.3) {Steuerung};
    \node[above of=Verar] (PC) {Simulationsumgebung};
    
    \draw[myarrow] (BeaconSig.east) -- ++(2,0) -- ++(0,0.2) |- (Verar.170);
	\draw[myarrow] (IMU.east) -- ++(2.5,0) -- ++(0,0.5) |- (Verar.180);
	\draw[myarrow] (Pos.110) -- ++(0,3.45) -- ++(2,0) -- ++(0,1) |- (Verar.190);
	\draw[myarrow] (Steuer.west) -- ++(-4,0) -| (Pos.70);

    \begin{pgfonlayer}{background}
        \path (App.north -| BeaconSig.west)+(-0.2,0.2) node (a) {};
        \path (IMU.south -| BeaconSig.east)+(+0.2,-0.2) node (b) {};
        \path[fill=blue!10,rounded corners, draw=black!50, dashed] (a) rectangle (b);
        
        \path (Pos.north -| Robo.west)+(-0.2,0.2) node (a) {};
        \path (Robo.south -| Robo.east)+(+0.2,-0.2) node (b) {};
        \path[fill=blue!10,rounded corners, draw=black!50, dashed] (a) rectangle (b);
        
        \path (PC.north -| PC.west)+(-0.2,0.2) node (a) {};
        \path (Steuer.south -| PC.east)+(+0.2,-0.2) node (b) {};
        \path[fill=blue!10,rounded corners, draw=black!50, dashed] (a) rectangle (b);    
    \end{pgfonlayer}
\end{tikzpicture}
\caption{Überblick auf die Software-Anforderungen}
\label{fig:SoftArch}
\end{figure}
\tikzstyle{sensor}=[draw, fill=yellow!20, text width=4cm, text centered, minimum height=2.5em]
\begin{figure}[b!]
\centering
\begin{tikzpicture}
	\tikzset{
    	myarrow/.style={->, >=latex', shorten >=1pt, thick}, 
	}  
    \node[sensor] (Grund) at (0,3) {Grundriss-Analyse};
    \node[sensor] (Pos) at (0,0) {Positionierung der Beacons \\ manuell/optimiert};
    \node[sensor] (SimSim) at (0,-3) {Simulation};
    \node[above of=Grund] (Simu) {Simulationsumgebung};
    
    \draw[myarrow] (Grund.south) to (Pos.north);
	\draw[myarrow] (Pos.south) to (SimSim.north);

    \begin{pgfonlayer}{background}    
        \path (Simu.north -| Simu.west)+(-0.2,0.2) node (a) {};
        \path (SimSim.south -| Simu.east)+(+0.2,-0.2) node (b) {};
        \path[fill=red!10,rounded corners, draw=black!50] (a) rectangle (b);    
    \end{pgfonlayer}
\end{tikzpicture}
\caption{Überblick auf die Software-Anforderungen}
\label{fig:SimArch}
\end{figure}