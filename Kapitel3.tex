\chapter{Simulation}
Hier steht dann Kapitel 3.

\section{Technische/Physiklische Vorgänge}
Vielleicht kann man hier auch schon abstrakte Modelle von Signalausdehnung (z.B. Extended Hata Modell - siehe ), Reflektion und Absorbtion annehmen. Für die Erklärung im zweiten Kapitel benötige ich auch die Störung und Interferenzen gleicher Signale. Ich muss ja schließlich erklären, warum ich nicht viele viele iBeacons nebeneinander klatschen kann, um eine möglichst genaue Standortbestimmung zu erhalten (denn die Signale stören sich untereinander).  Desweiteren der Smartphone-Aufbau (hier eingehen auf die Antenne vom Smartphone), iBeacon-Aufbau und Funktionsweise und schlussendlich auch den MIRA-Roboter und den Youbot ein wenig erklären.// //

Erklärung für die Wahl des Parcours. Hier könnte dann drinne stehen, warum ich welche Situation erstellt habe. Mir ist z.B. aufgefallen, dass die Beacons sich gegenseitig in der Sendestärke beeinflussen. Deswegen musste ich Einzelmessungen vornehmen und hier ließe sich auch begründen nicht zu viele Beacons in der realen Anwendung zu plazieren, um durch Elektrosmog nicht zu viele Einflussfaktoren in die Lokalisierung miteinfließen zu lassen. Viel hilft hier nicht immer viel, weniger ist manchmal besser, etc.

Zuerst werde ich erklären, wie ich die Einzelmessungen und 

\section{Parameterschätzung}
\subsection{App}
\subsection{Youbot}


Graphik mit Leistungsverbrauch der beacons in Abghängigkeit mit ihren Einstellungen. Die Farbe soll dabei der Leistungsverbrauch und die Achsen die Intervallzeit und die Sendeleistung darstellen. Hier kann man sich auch überlegen die theoretische Akkulaufzeit mit anzugeben. Die Knopfbatterien sind vom Typ CR2450 mit einem Energiegehalt von ca. 1,8 Wh \cite{CR2450}. Der Mikrocontroller kann hier nicht mit eingerehnet werden, da lediglich ein Verbrauch in Abhängigkeit mit der Prozessorgeschwindigkeit \cite{ARM} gegeben ist und für die Beacons keine Informationen vorliegen, in welcher Taktfrequenz diese betrieben werden. Natürlich hat der Prozessor auch eine höhere Taktfrequenz, wenn die Sendeintervalle der Beacons häufiger stattfinden und somit ist beides aneinander gekoppelt. Deswegen Messung des Stromverbrauchs am Beacon bei minimaler Sendeleistung mit den unterschiedlichen Intervallen (niedrigster Stufe, mittlerer und höhster) und dann einfach linear approximieren. 

    