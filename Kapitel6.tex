\chapter{Zusammenfassung und Ausblick}

Hier noch einmal das Bild von dem Energieverbrauch, der Richterskala, dem Lighthouse Keeper und von der Evaluation zeigen. Betonen, dass es sich hier um Grundlagenforschung handelt und das das Modell noch um Isolinien erweitert werden kann. Es noch viel zu tun gibt bei den Beacons (könnten ja mal den 2,4 Ghz Bereich scannen, ob er frei ist). Das man immer abwägen muss zwischen Genauigkeit und Reichweite der Lokalisierung und Wartungsaufwand. Das die Modelle so nicht eins zu eins übertragbar sind und mein verwendeter Lokalisierungsalgorithmus nur rudimentär ist. Demzufolge kann mit besseren Algorithmen die Qualität der Lokalisierung weiter verbessert werden. Jedoch würde sich dadurch nichts am Optimierungsalgorithmus ändern, sodass er eine gute Grundlage zur automatisierten Konfiguration bildet. Zumindest bietet er gute Positionen für Beacons. 

Ich hoffe, dass diese Arbeit einen Teil dazu beiträgt, die Beacon-Technologie weiter voran zu bringen. 