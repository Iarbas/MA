\documentclass[12pt,a4paper,twoside,BCOR=8mm,openright]{scrbook}
\usepackage{pdflscape}                         % fuer Querformate und so
\usepackage{wrapfig}                           % ermoeglicht, Bilder von Text umfliessen zu lassen
\usepackage[pdftex]{graphicx}                  % einbinden Graphiken
\usepackage[T1]{fontenc}                       % Deutschzeug
\usepackage[utf8]{inputenc}                    % für die Kodierungsangabe des Editors
%\usepackage[marvosym]                         %
\usepackage{ngerman}                           % 
\usepackage{mathrsfs,amssymb}                  % Mathezeug
\usepackage[intlimits]{empheq}                 %
\usepackage{amsmath}                           %
\usepackage{array, ragged2e}                   % Tabellenerweiterung ->
\usepackage{tabularx}
\usepackage{colortbl}						   % Farbe in Tabellen
\usepackage{multirow}						   % Selbsterklaerend
\usepackage{rotating}						   % <-
\usepackage{ifthen,calc}                       % Verbesserte Zuweisungen und Kontrollstrukturen
\usepackage{pifont,mathptmx,charter,courier}   % Schriften und Symbole
\usepackage[scaled]{helvet}                    %
%\usepackage{natbib}
\usepackage[numbers,square]{natbib}            % fuer den Bibtex, damit plaindin verwendet werden kann
%\usepackage{caption}                           % fuer die Beschriftung von Grafiken
\usepackage[format=hang]{caption}
\captionsetup[wrapfigure]{format=plain}		   % auch unter Abb. schreiben
\captionsetup[figure]{format=plain}
\usepackage{float}                             % macht die figures dorthin, wohin ich sie haben moechte
\usepackage{url}
\PassOptionsToPackage{hyphens}{url}
\usepackage[colorlinks=false]{hyperref}
\usepackage{breakurl}
\expandafter\def\expandafter\UrlBreaks\expandafter{\UrlBreaks%
  \do\a\do\b\do\c\do\d\do\e\do\f\do\g\do\h\do\i\do\j%
  \do\k\do\l\do\m\do\n\do\o\do\p\do\q\do\r\do\s\do\t%
  \do\u\do\v\do\w\do\x\do\y\do\z\do\A\do\B\do\C\do\D%
  \do\E\do\F\do\G\do\H\do\I\do\J\do\K\do\L\do\M\do\N%
  \do\O\do\P\do\Q\do\R\do\S\do\T\do\U\do\V\do\W\do\X%
  \do\Y\do\Z\do\*\do\-\do\~\do\'\do\"\do\-}%
\usepackage{acronym}						   % fuer die Abkuerzungen
\usepackage{icomma}

\usepackage{tikz}							   % fuer schoene Zeichnungen
\usetikzlibrary{intersections, calc, arrows, shapes, shapes.symbols, positioning, shadows, shadows.blur, trees, decorations.markings, patterns, pgfplots.groupplots, fadings}
\usepackage{smartdiagram}
\usepackage{pgfplots}
\usepgfplotslibrary{colormaps}
\pgfplotsset{compat=1.6, /pgf/number format/use comma, /pgf/number format/1000 sep={}, y tick label style={/pgf/number format/use comma, /pgf/number format/1000 sep={}}, x tick label style={/pgf/number format/use comma, /pgf/number format/1000 sep={}}}


% Zum Zeichnen von Kreisdiagrammen
\tikzset{
  arc/ccw/.initial=1,
  arc/large/.initial=0,
  arc ccw/.style={/tikz/arc/ccw=1},
  arc cw/.style={/tikz/arc/ccw=0},
  arc large/.style={/tikz/arc/large=1},
  arc small/.style={/tikz/arc/large=0}
}
\tikzset{
  arc to/.style={
    to path={
      let \p{arcto@1}=($(\tikztotarget)-(\tikztostart)$),
          \n{arcto@a}={acos(.5*veclen(\p{arcto@1})/(#1))},
          \n{arcto@s}={180 + atan2(\p{arcto@1}) - ifthenelse(\pgfkeysvalueof{/tikz/arc/large}==1,-1,1)*\n{arcto@a} + ifthenelse(\pgfkeysvalueof{/tikz/arc/ccw}==1,180,0)},
          \n{arcto@e}={      atan2(\p{arcto@1}) + ifthenelse(\pgfkeysvalueof{/tikz/arc/large}==1,-1,1)*\n{arcto@a} + ifthenelse(\pgfkeysvalueof{/tikz/arc/ccw}==1,180,0)}
      in
       \if\pgfkeysvalueof{/tikz/arc/ccw}0
         (\tikztostart) arc [start angle=\n{arcto@s}, end angle=\n{arcto@e}, radius={#1}] \tikztonodes
       \else
         (\tikztostart) arc [start angle=\n{arcto@e}, end angle=\n{arcto@s}, radius={#1}] \tikztonodes
       \fi
    }
  }
}
\definecolor{dblue}{rgb}{0, 0, 0.56}
\definecolor{ice}{rgb}{0.3, 0.75, 0.93}
\definecolor{mint}{rgb}{0.8, 0.8, 0.8}

% Zur Invertierung von Colormaps
\makeatletter
\def\customrevertcolormap#1{%
    \pgfplotsarraycopy{pgfpl@cm@#1}\to{custom@COPY}%
    \c@pgf@counta=0
    \c@pgf@countb=\pgfplotsarraysizeof{custom@COPY}\relax
    \c@pgf@countd=\c@pgf@countb
    \advance\c@pgf@countd by-1 %
    \pgfutil@loop
    \ifnum\c@pgf@counta<\c@pgf@countb
        \pgfplotsarrayselect{\c@pgf@counta}\of{custom@COPY}\to\pgfplots@loc@TMPa
        \pgfplotsarrayletentry\c@pgf@countd\of{pgfpl@cm@#1}=\pgfplots@loc@TMPa
        \advance\c@pgf@counta by1 %
        \advance\c@pgf@countd by-1 %
    \pgfutil@repeat
%\pgfplots@colormap@showdebuginfofor{#1}%
}%

\makeatother

% Definition eines Blockes
\tikzset{
    block/.style={draw, rectangle, rounded corners, very thick, minimum height=1cm, text centered,},
     block2/.style={draw, fill=blue!20, rectangle, rounded corners, text width=2.9cm, text centered, minimum height=1.5cm},
    sum/.style={draw, fill=blue!20, circle, node distance=1cm,minimum size=0.6cm},
}

% siehe eine Zeile drunter
\usepackage{accents}                           
\renewcommand{\dot}[1]{\accentset{\mbox{\large .}}{#1}} % fuer einen besseren Punkt bei Zeitableitungen
\renewcommand{\ddot}[1]{\accentset{\mbox{\large .\hspace{-0.2ex}.}}{#1}}

% Einstellungen
\usepackage{parskip}                                 % Einzug aus, Absatzabstand ein
\usepackage{pdfpages}                                % ermoeglicht pdf-Seiten einzufuegen
\usepackage{ulem}                                    % neue Strich-Funktionen (z.B. Text durchstreichen)
\tolerance=2000                                      % Toleranz für Wortzwischenräume
\setlength{\emergencystretch}{20pt}                  % Zusätliche Zeilendehnbarkeit
\setlength{\fboxrule}{1pt}\setlength{\fboxsep}{4mm}  % Rechtecke um Gleichungen

% Zur Veraenderung vom Zeilenabstand und der Ueberschriften
\usepackage{setspace}                          % fuer den Zeilenabstand
\renewcommand{\baselinestretch}{1.25}\normalsize % Zeilenabstand
\usepackage{titlesec}                          % zur Formatierung der Ueberschriften
%\titleformat{\chapter}[hang]{\normalfont\Large\bfseries}{\thechapter}{0.5em}{}          % Format muss beim chapter mit rein, sonst geht es nicht.
%\titleformat{\section}[hang]{\normalfont\large\bfseries}{\thesection}{0.5em}{}
%\titleformat{\subsection}[hang]{\normalfont\bfseries}{\thesubsection}{0.5em}{}
\titlespacing*{\chapter}{0pc}{20pt}{10pt}[0pc]
\renewcommand*\chapterheadstartvskip{\vspace*{-1cm}}
\titlespacing{\section}{0pc}{12pt}{5pt}[0pc]
\titlespacing{\subsection}{0pc}{10pt}{2pt}[0pc]

\usepackage[a4paper, right= 2.5cm, textwidth=16cm, textheight=23.7cm]{geometry}
% Hier drunter ist das Zeug von Andreas
%\setlength\voffset{-1in}                             
%\setlength\hoffset{-1in}
%\setlength\topmargin{2cm}
%\setlength\textheight{23.7cm}
%\setlength\textwidth{16.001cm}
%\setlength\footskip{1cm}
%\setlength\headheight{0.6cm}
%\setlength\headsep{1cm}
%\setlength{\skip\footins}{0.119cm}
%\setlength\tabcolsep{1mm}
%\renewcommand\arraystretch{1.3}
%\renewcommand{\footnoterule}{\rule{200pt}{1pt}{\vspace*{0.5cm}}}

\renewcommand{\figurename}{Abb.} 				% Abkuerzung von Abbildung

\usepackage{fancyhdr}                          % Einstellungen für Kopf- und Fusszeilen
\pagestyle{fancy}
\fancyhf{}
\fancyhead[RO,LE]{\thepage}                    % ungerade rechts, gerade links
\renewcommand{\chaptermark}[1]{\markboth{#1}{}}% definiert den Befehl chaptermark in leftmark neu, sodass nur der Name angezeigt wird
\fancyhead[C]{\nouppercase{\leftmark}}

% Definition einiger Farben
\definecolor{airforceblue}{rgb}{0.36, 0.54, 0.66}
\definecolor{ballblue}{rgb}{0.13, 0.67, 0.8}

% Fuer die Zentrierung von Tabelleneintraegen
\newcolumntype{C}[1]{>{\centering\let\newline\\\arraybackslash\hspace{0pt}}m{#1}}
\newcolumntype{M}[1]{>{\RaggedRight\arraybackslash}m{#1}}