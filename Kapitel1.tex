\chapter{Einleitung}
Hier steht dann die Einleitung.

Hier könnten theoretische Annahmen beschrieben und erklärt werden. Zum Beispiel ließe sich hier die Funktionsweise von iBeacons erklären, zusammen mit einer kleinen Einleitung in WLAN und Bluetooth (also welche Kanäle benutzt Bluetooth LE, sind die Beacons so schlau und suchen vor einer Signalsendung den Raum ab, ob gerade jemand anderes sendet und warten somit auf einen "freien" Raum). Desweiteren warum 2,4 Ghz als Frequenz beim Bluetooth LE-Format verwendet wird (Politik mit Vergabe der Sendefrequenzen, bei 2,4 Ghz ist die Eigenschwingung von Wasser, deswegen ist diese Frequenz frei, usw.). Vielleicht kann man hier auch schon abstrakte Modelle von Signalausdehnung (z.B. Extended Hata Modell - siehe ), Reflektion und Absorbtion annehmen. Für die Erklärung im zweiten Kapitel benötige ich auch die Störung und Interferenzen gleicher Signale. Ich muss ja schließlich erklären, warum ich nicht viele viele iBeacons nebeneinander klatschen kann, um eine möglichst genaue Standortbestimmung zu erhalten (denn die Signale stören sich untereinander).  Desweiteren der Smartphone-Aufbau (hier eingehen auf die Antenne vom Smartphone), iBeacon-Aufbau und Funktionsweise und schlussendlich auch den MIRA-Roboter und den Youbot ein wenig erklären.// //

Als Idee: Zuerst werden die Grundlagen der Hardware, dann der Software erklärt. Anschließend kommen die mathematischen Modelle für Ausbreitung von WLAN/Bluetoooth-Signalen in Abhängigkeit zu ihrer eingestellten Stärke. 