\chapter{Einleitung}
\begin{wrapfigure}{r}{6.5cm}
\centering
\includegraphics[scale=0.8]{Bilder/GPS.png} 
\caption{Skizze satellitengestütztes Lokalisierungssystem \cite{GPS}}
\label{GPS}
\end{wrapfigure}
Herkömmliche Navigationssysteme für den "`Outdoor"'-Bereich haben viele Bereiche des alltäglichen Lebens weitgehend erleichtert. Zurück liegen die Tage, in denen Fahrtstrecken über unbekannte Straßen und durch fremde Länder mit analogen Karten lange geplant wurden. Die Navigation, bzw. Lokalisierung basiert dabei auf einer Positionsbestimmung mithilfe von Funksignalen, die von Satelliten ausgesendet werden. Aus diesen Signalen können die Navigationssysteme ihre Position auf der Erde berechnen und schließlich die Information dem Nutzer zur Verfügung stellen. Die eigene Lokalisierung wäre nun auch für den Nutzer in einem Gebäuden hilfreich, damit er sich orientieren kann und beispielsweise in einem großen Kaufhaus schneller zu einem Geschäft findet. Um die Vorteile einer Lokalisierung im "`Indoor"'-Bereich anzuwenden, können die bisherigen satellitengestützen Systeme (GPS, GLONASS, Galileo) jedoch nicht genutzt werden, da die Strukturen der Gebäude ihre Signale absorbieren, bzw. reflektieren. Seit einigen Jahren arbeiten Forschungseinrichtungen und Unternehmen an einer Lösung für die Lokalsierung von Objekten in Gebäuden. Die Forschungsprojekte verfolgen dabei verschiedene Ansätze und Technologien um dieses Ziel zu erreichen, wie z.B. die Projekte "`EVARILOS"' \cite{EVA}, "`Google Indoor Maps"' \cite{GIM} und "`Mobile Indoor Localization"' \cite{MIL}. Jedoch existiert noch keine Lösung die sich bereits durchgesetzt hat und komerziell nutzbar wäre.\\ \\
Eine neue und vielversprechende Technologie zur innerräumlichen Lokalisierung stellt dabei die Entwicklung von iBeacons (z. Dt. Leuchtfeuer) dar. Die Vorteile dieses Systems gegenüber bisherigen Lösungen sind zum einen die geringen Kosten, sowie die hohe Flexibilität und Autonomie der einzelnen Elemente. Viele bisherigen Technologien zur Indoor-Lokalisierung, wie z.B. in den Boden eingelassene künstlichen Magnetfeldern \cite{Magnet}, RFID-Transponder \cite{RFID} oder funkbasierten Lösungen\citep{WLAN} sind zwar erprobt und erzielen eine hohe Lokalisierungsgenauigkeit, besitzen jedoch den Nachteil einer teuren und aufwendigen Infrastruktur, welche meist auch eine bauliche Änderung am Gebäude benötigt. Die Einfachheit der Anbringung der Beacons, die Kostenvorteile und die weitere Verbreitung des verwendeten Bluetooth-Protokolls auf mobilen Geräten sorgen dabei für ein weites Einsatzspektrum. Jedoch der größte Vorteil der iBeacons gegenüber den konkurrierenden Technologien ist die variable Verwendung der Beacons für beispielsweise einfache, rudimentäre Positionsbestimmungen bis hin zur hochpräzisen Ortung. Je nach Einsatzszenario könnten die Beacons problemlos angepasst und so für ihren Zweck hin optimiert werden.\\ \\
Um mit den Beacons ein Lokalisierungssystem in Gebäuden aufzubauen, ähnlich dem im Outdoor-Bereich, benötigt es geeignete Verfahren die Beacons zu positionieren und entsprechend auf ihren Einsatzzweck anzupassen. Während für die Satelliten der Outdoor-Systeme Modelle vorhanden sind, die deren Anzahl, Flughöhe und Sendeleistung für eine bestmögliche Lokalisierung berechnen, fehlen die Konzepte für den Indoor-Bereich. Zudem mangelt es an Verfahren die Gebäude mit Beacon-Systemen zweifelsfrei auf die Qualität der Lokalisierung und der räumlichen Abdeckung hin zu validieren. Gegenstand dieser Arbeit soll es dabei sein, ein geeignetes Konzept für die Erstellung von Beacon-Konfigurationen zu gestalten und die Konfiguration anschließend experimentell zu validieren. Dabei soll auf ein automatisiertes und strukturiertes Verfahren mithilfe von Robotern zurückgegriffen werden, um die Validierung weniger fehleranfällig zu designen und somit zu standardisieren.\\ \\
Aus diesen Anforderungen stellen sich drei zentrale Zielstellungen:
\begin{itemize}
\item Entwicklung eines Frameworks und einer systematischen Versuchsplanung für die Messung von Beacon-Signalen
\item Auswahl und Parameterbestimmung eines Modells für die Signalausbreitung von Beacons mit anschließender Erstellung einer Simulationsumgebung, sowie eine sich daraus konkludierenden optimalen Verteilung von Beacons in einem Raum 
\item Validierung des Modells und der Simulationergebnisse in einem Testszenario mithilfe eines Roboters  
\end{itemize}
Um die Problematik eines bisher fehlenden Planungskonzeptes darzulegen, wird in zuerst der Stand der Technik von Beacons beschrieben und erläutert. Daraus schlussfolgern sich ersten Ansätze, wie ein mögliches Verfahren zur Beacon-Konfiguration aufgebaut sein muss. In Kapitel 2 werden für die erste Zielsetzung alle Werkzeuge zur Messung der Beacon-Signale vorgestellt und in Kapitel 3 diese ausgewertet. Mit den Messwerten werden anschließend die Parameter für die Signalausbreitung der Beacons in einem freien Raum bestimmt und damit die Grundlage für ein Simulationsprogramm gelegt. Da es keine Standards oder Richtlinien für die Lokalisierungsgenauigkeit gibt, werden zusätzlich aus den Messwerten eine Richterskala bestimmt und diese erklärt. Mithilfe der Simulation und eines Optimierungsalgorithmus werden anhand der selbstdefinierten Richterskala geeignete Beacon-Konfigurationen berechnet und diese experimentell in Kapitel 4 überprüft. Dabei wird besonders auf Art der Validierung eingegangen, da sie automatisiert und standardisiert von einem Roboter durchgeführt wird. Am Ende der Arbeit werden die drei gesetzten Ziele mit dem Erreichten verglichen, die Ergebnisse daraus diskutiert und anschließend wird das Konzept daran bewertet. 
 





% Bietet sich leider hier nicht an:(

%Zur Anschaulichkeit sind nachfolgend in Abbildung ... alle funkbasierte Verfahren zur Entfernungsmessung einmal vorgestellt.
%
%\begin{tikzpicture}[level 1/.style={sibling distance=40mm}, edge from parent/.style={->,draw}, >=latex]
%% root of the the initial tree, level 1
%\node[root] {HF-basierende Entfernungsmessungen}
%% The first level, as children of the initial tree
%  child {node[level 2] (c1) {Defining node and arrow styles}}
%  child {node[level 2] (c2) {Positioning the nodes}}
%  child {node[level 2] (c3) {Drawing arrows between nodes}};
%
%% The second level, relatively positioned nodes
%\begin{scope}[every node/.style={level 3}]
%\node [below of = c1, xshift=15pt] (c11) {Setting shape};
%\node [below of = c11] (c12) {Choosing color};
%\node [below of = c12] (c13) {Adding shading};
%
%\node [below of = c2, xshift=15pt] (c21) {Using a Matrix};
%\node [below of = c21] (c22) {Relatively};
%\node [below of = c22] (c23) {Absolutely};
%\node [below of = c23] (c24) {Using overlays};
%
%\node [below of = c3, xshift=15pt] (c31) {Default arrows};
%\node [below of = c31] (c32) {Arrow library};
%\node [below of = c32] (c33) {Resizing tips};
%\node [below of = c33] (c34) {Shortening};System
%\node [below of = c34] (c35) {Bending};
%\end{scope}
%
%% lines from each level 1 node to every one of its "children"
%\foreach \value in {1,2,3}
%  \draw[->] (c1.195) |- (c1\value.west);
%
%\foreach \value in {1,...,4}
%  \draw[->] (c2.195) |- (c2\value.west);
%
%\foreach \value in {1,...,5}
%  \draw[->] (c3.195) |- (c3\value.west);
%\end{tikzpicture} 
%
%\cite{IndLoc} und \cite{DisLok}
