\chapter{Stand der Technik -- Einführung in die Beacon-Technologie}
Der erste Teil dieses Kapitels beschäftigt sich mit der Entstehungsgeschichte von Beacons und nachfolgend mit deren Funktionsweise. Hier wird besonders auf die technischen Möglichkeiten der Technologie eingegangen. Im letzten Abschnitt wird erläutert, wo Beacon-Systeme zum Einsatz kommen und wie aktuelle Lösungsansätze für die Planung der dafür nötigen Infrastruktur aufgebaut sind. Anschließend wird aus den gegebenen Charakteristika der Beacons und der momentan genutzten Fähigkeiten differenziert und eine Aussicht auf ein zukünftiges Konzept zu einer effizienteren Nutzung gegeben.
\section{Entwicklungsgeschichte}
Den Grundstein für die Beacon-Technologie legte das finnische Unternehmen Nokia im Jahre 2006. Damals entwickelte die Firma den neuen Standard \textit{Wibree} für die Funkübertragung, der den veralteten Bluetooth-Standard ersetzen sollte. Mit der Neuentwicklung versprach man sich im Gegensatz zu Bluetooth einen geringeren Stromverbrauch und geringere Produktionskosten bei gleichbleibenden Übertragungsraten. Ab dem Jahr 2009 wurde der Bluetooth-Standard um Wibree ergänzt und erst unter den Bezeichnungen \textit{Ultra Low Power Bluetooth} (ULP) und dann später als \textit{Bluetooth Low Energy} (BLE) darin aufgenommen \cite{Wib2BLE} und anschließend als \textit{Bluetooth Smart} vermarktet. Da viele Hersteller von mobilen Geräten in ihren Datenblättern die Unterstützung von BLE nicht explizit erwähnen, findet sich meist folgendes Logo \ref{BLElogo} in den Produktbeschreibungen und das erkennen lässt, ob die Geräte den neuen Standard unterstützen. \\ \\
Die Idee der Nutzung von Bluetooth Low Energy zur Indoor-Lokalisierung stammt dabei von der Firma Apple Inc. und wurde von ihr im Jahre 2013 auf der WWDC (Worldwide Developers Conference)\cite{Apple} unter dem Namen \textit{iBeacon} angekündigt. Obwohl zu dem Zeitpunkt noch kein fertiges Gerät zur Verfügung stand, wurde diese Technologie als Neuerung in Apples mobilem Betriebssystem iOS 7 vorgestellt. Jedoch verzichtet Apple seither auf die Produktion von iBeacons, was andere Unternehmen nutzen, um selbst in den Markt einzusteigen. Deren Produkte wurden darauf in Beacons umbenannt und unterstützen zusätzlich die mobilen Betriebssysteme Android ab Version 4.3, Windows Phone 8 und die neueste Version von Blackberries OS \cite{Bibel}. Somit wäre die Beacon-Technologie mit der nötigen Hardware-Unterstützung in mittlerweile über 99,5\% aller mobilen Geräte (Smartphones, Tablets, Smartwatches etc.) weltweit nutzbar \cite{MobGerSt}. \par\bigskip
\begin{figure}[H]
\centering
\includegraphics[scale=0.07]{Bilder/BLE.png} 
\caption{Offizielles Bluetooth Smart Logo \cite{BLElogo}}
\label{BLElogo}
\end{figure}
%\begin{figure}[H]
%\centering
%\includegraphics[scale=0.5]{Bilder/iBeaconLogo.png} 
%\caption{Offizielles iBeacon Logo \cite{iLogo}}
%\label{iLogo}
%\end{figure}
%
\section{Aufbau und Funktionsweise von Beacons} 
\begin{wrapfigure}{r}{6cm}
\begin{flushright}
\includegraphics[scale=0.07]{Bilder/BeaconSchicht.png}
\caption{Explosionszeichnung Beacon \cite{BeaEx}}
\label{fig:BeaEx}
\begin{picture}(0,0)
\put(-150,177){Schutzhülle}
\put(-85,182){\line(1,0){25}}
\put(-150,137){Platine}
\put(-110,142){\line(1,0){50}}
\put(-150,100){Batterie}
\put(-100,105){\line(1,0){40}}
\put(-150,60){Silikonplatte}
\put(-80,65){\line(1,0){20}}
\end{picture}
\end{flushright}
\end{wrapfigure}
Die Grundbausteine der kleinen Leuchtfeuer sind in der rechten Abbildung \ref{fig:BeaEx} ersichtlich, in der ein \textit{Estimote Beacon} der Firma Estimote Inc. in seine einzelnen Bestandteile untergliedert ist. Ein Beacon misst ungefähr 5,5 cm $\times$ 3,5 cm $\times$ 1,5 cm in Länge, Breite und Höhe und ist dabei 50 g schwer. Zur Anbringung der Beacons an Wänden, Decken usw. dient die auf der Rückseite befindliche Silikonplatte. Diese haftet an nahezu jeder glatten Oberfläche und kann mithilfe von Wasser sehr einfach gereinigt werden. Somit können die Beacons im Grunde unzählige Male angebracht und wieder abgenommen werden. Die äußere Schutzhülle besteht dabei ebenfalls aus einem Silikon und schützt die inneren Bauteile. Zu den inneren Komponenten gehören die Platine mit dem darauf verlöteten Nordic nRF51822 Chip (32-Bit ARM Cortex M0 CPU mit 256 kB Flash-Speicher und dem 2,4 GHz BLE-Sendemodul) \cite{nRF5}, eine Antenne und eine Knopfbatterie zur autarken Stromversorgung. Die Beacons haben eine variable Sendeleistung von 4 dBm bis -30 dBm (entspricht einer Leistung von 2,512 Milliwatt bis 1 Mikrowatt) und übertragen ihre Daten in Intervallen von 50 bis 0,5 Hertz. Die Kommunikation verläuft dabei bidirektional, d.h. vom Beacon zum Empfangsgerät und zurück. Während die Kommunikation eines Beacons zu einem mobilen Gerät dazu dient, die Lokalisierung des Gerätes zu ermöglichen, eignet sich die Kommunikation von einem Smartphone, Tablet etc. zu einem Beacon zu dessen Programmierung und zur Überprüfung des Betriebszustandes, wie z.B. dem Akku-Ladezustand. \\ \\
Die bei der Signalübertragung verwendete Protokoll-Architektur Bluetooth Low Energy sendet dabei im 2,4 Ghz Band, welches ebenfalls von den Protokollen 802.11 ac/a/b/g/n, älteren Bluetooth-Standards, ZigBee, NFC, etc. genutzt wird. Eine kompakte Übersicht zu den einzelnen Varianten und deren Eigenschaften findet sich im Anhang in Bild \ref{fig:FUE}. Die Vorteile des 2,4 Ghz Bandes gegenüber anderen Frequenzbändern ergeben sich aus einer großen Reichweite der Funksignale und einer geringen Größe der Antenne für deren Erzeugung. Jedoch besitzt dieses Band auch gewisse Nachteile, die bei der Auslegung von Beacon-Konfigurationen beachtet werden müssen. Die Verwendung des Protokolls im 2,4 Ghz Band ist dabei historisch bedingt und nahezu alternativlos, da es zu den ISM-Bändern \cite{BuNet} zählt und nur diese somit frei nutzbar sind. Bei vermehrter Nutzung des 2,4 Ghz Bandes für die Datenübertragungen in WLAN oder Bluetooth können geringfügige Störungen in den Übertragungen auftreten. Somit hängt die Qualität der Signale und schlussendlich auch die Lokalisierungsgenauigkeit von der lokalen Auslastung des 2,4 Ghz Bandes ab. Des Weiteren können auch sich zwischen Sender und Empfänger befindliche physische Objekte einen störenden Faktor auf die Empfangsqualität ausüben. In den Bildern \ref{fig:Wand} und \ref{fig:Hand} ist eine mögliche Beeinträchtigung der Empfangsqualität von BLE-Signalen durch eine im Sichtfeld befindliche Wand oder durch einen menschlichen Körper dargestellt. Denn durch die physikalischen Eigenschaften des 2,4 Ghz Bandes werden die Funksignale durch Materialien wie Wasser und Stahl besonders gut absorbiert. Die meisten Gebäude wurden aus Stahlbeton errichtet und der menschliche Körper besteht aus einem großen Anteil aus Wasser. Diese Umstände wirken sich besonders negativ auf die Qualität der BLE-Signale aus und eignen sich dadurch eigentlich nicht für die Anwendung einer Lokalisierung von Menschen in Gebäuden. Jedoch sei dies nur am Rande erwähnt und würde unter zusätzlicher Betrachtung dieser Aspekte den zeitlichen Rahmen dieser Arbeit in großem Ausmaß sprengen. Zusammenfassend werden hier noch einmal die Gründe für eine Qualitätsminderung der Lokalisierung festgehalten: die Schwächung der Signale an Objekten, die Phasenauslöschung durch andere Signalquellen und unberechenbare Reflexionen.\\ \\
\begin{figure}[H]
$\begin{minipage}[b]{7cm}
\centering
\includegraphics[scale=0.13]{Bilder/Wand.png} 
\caption{Signalschwächung durch Objekte, z.B. Wände \cite{GSwiB}}
\label{fig:Wand}
\end{minipage}
\hspace{3cm}
\begin{minipage}[b]{7cm}
\centering
\includegraphics[scale=0.13]{Bilder/Hand.png} 
\caption{Menschliche Körper blockieren zusätzlich die Signale \cite{GSwiB}}
\label{fig:Hand}
\end{minipage}$
\end{figure}
Die Daten, die bei der Kommunikation übertragen werden, sind dabei essentiell für die Positionsbestimmung von Objekten und bestehen aus einer:
\begin{itemize}
\item Identifikationsnummer (Länge von 16 Bytes),
\item eingestellten Sendeleistung (2 Bytes)
\item und zusätzlichen Information, beschrieben als Major und Minor (jeweils 2 Bytes).
\end{itemize}
Mithilfe geeigneter Software lassen sich diese Parameter ändern und den Anforderungen entsprechend anpassen. Zusätzlich lässt sich noch die Intervalllänge der Sendefrequenz variieren. Im Falle der Estimote Beacons geschieht dies mit der kostenlosen Software "`Estimote"' und ist für die Plattformen iOS und Android auf mobilen Geräten verfügbar. Mithilfe dieses Werkzeuges kann die Ortungsgenauigkeit signifikant erhöht bzw. verringert werden. Da die genauen Einflüsse dieser Parameter auf die Batterielebensdauer und die Ortungsgenauigkeit noch nicht erforscht wurden oder zumindest nicht öffentlich zugänglich sind, werden diese in Kapitel 4 untersucht und erklärt. Denn gerade der adaptive Aspekt für die Indoor-Lokalisierung lässt diese Technologie so attraktiv erscheinen.
\section{Ortungsmethoden}
Mit der Identifikationsnummer kann zwischen den einzelnen Beacons unterschieden werden, wodurch erst die Möglichkeit einer Lokalisierung entsteht. Denn die Technik setzt voraus, dass die Identitäten der Beacons in einer Datenbank mit Positionsangabe in einem Gebäude hinterlegt sind. Beim Empfang eines gültigen Signals wird dder Sender in der Datenbank gesucht und anschließend dessen Distanz zum Empfangsgerät geschätzt. Die Berechnung findet auf der Grundlage eines Ausbreitungsmodells von Signalen in Abhängigkeit zur eingestellten Sendeleistung des Beacons statt, in der die empfangene Signalstärke als RSSI-Wert interpretiert und dadurch eine Distanz geschätzt wird. Die Modelle der Hersteller sind typischerweise nicht frei zugänglich, weswegen sie auch hier nicht vorgestellt werden. Die als zusätzliche Informationen gekennzeichneten Daten sind eine Erweiterung der Identifikationsnummer und bieten lediglich einen gesteigerten Komfort für die Entwicklung der Datenbanken. Im nächsten Abschnitt wird dies in Tabelle \ref{table:BeBe} noch einmal veranschaulicht.\\
\begin{table}[H]
\centering
\begin{tabular}{|c|c|C{3cm}|C{3cm}|C{3cm}|}
\hline
\rowcolor{airforceblue} \multicolumn{2}{|c|}{\textbf{Geschäftsstandort}} & \textbf{Berlin}  & \textbf{Magdeburg} & \textbf{München} \\ \hline
\multicolumn{2}{|c|}{\cellcolor{ballblue} \textbf{UUID}}    & \multicolumn{3}{c|}{U8T7V56I-4689-10U9-7G63B4GAR21M}\\ \hline
\multicolumn{2}{|c|}{\cellcolor{ballblue} \textbf{Sendeleistung} }    & -12 dBm   & -6 dBm & 1 dBm \\ \hline
\multicolumn{2}{|c|}{\cellcolor{ballblue} \textbf{Major}}    & 1  & 2 & 3\\ \hline
\cellcolor{ballblue} & \cellcolor{ballblue} \textbf{Kleidung}    & 10  & 10 & 10\\ \cline{2-5}
\cellcolor{ballblue} & \cellcolor{ballblue} \textbf{Elektronik}    & 20  & 20 & 20\\ \cline{2-5}
\cellcolor{ballblue} \multirow{-3}{*}{\textbf{Minor}}& \cellcolor{ballblue} \textbf{Küche}    & 30  & 30 & 30\\ \cline{2-5}
\hline
\end{tabular}
\caption{Beispiel der Informationsnutzung; in Anlehnung an: \cite{GSwiB}}
\label{table:BeBe}
\end{table}
Bei dem obigen Beispiel nutzt eine Warenhauskette mit mehreren Geschäftsstandorten die Beacon-Technologie zur Lokalisierung ihrer Besucher an ihren verschiedenen Standorten. Zur Vereinfachung besitzen die Beacons einen \textit{Universally Unique Identifier} (UUID) und unterscheiden sich jeweils nur in ihren zusätzlichen Informationen und der Sendeleistung. Die gezeigten Informationen liegen auf den mobilen Geräten der Nutzer vor, genauso wie eine Applikation auf den Geräten vorhanden sein muss, die diese Daten verarbeiten kann. Die Information Major steht dabei für den jeweiligen Standort und Minor für die Abteilung, in der die Signale der Beacons empfangen werden können und der RSSI-Wert in einem definierten Bereich liegt. Abhängig vom Konzept der Ortung kann dieses Wissen unterschiedlich genutzt werden. Bei der Beacon-Technologie unterscheidet man daher zwischen zwei Anwendungskonzepten der Ortung.  
\subsection{Definitionen verschiedener Anwendungsfelder}
Die einfachste Form einer Lokalisierung ist die Aufteilung eines Raumes in Bereiche, in der die Position eines Nutzers zu einem Beacon dadurch angegeben wird, ob er im Nah- oder Fernbereich zu ihm positioniert ist. Diese Art der Ortung ist dabei sehr ungenau, da die Distanzinformationen nicht explizit vorliegen, sondern nur ein homogener Distanz-Bereich um ein Beacon definiert ist. Die Distanz zu einer Funkbake wird dabei durch die empfangene Signalstärke von einem Beacon geschätzt. Diese Beurteilung beruht derzeit noch auf den Erfahrungen des Beacon-Programmierers bzw. des Installateurs der Beacon-Systeme. Als Beispiel für eine solche Definition wurden einmal vier Zustände in der Tabelle \ref{table:Ranging} festgelegt und deren Bereiche in Abhängigkeit zu der empfangenen Sendeleistung eines Beacons eingegrenzt. In einem fiktiven Einsatzszenario könnte somit ein Kunde in einem Supermarkt oder Warenhaus gezielt auf ein Sonderangebot in seiner Nähe aufmerksam gemacht oder zusätzliche Informationen zu einem Produkt in einer entsprechenden Applikation angezeigt werden.    
\begin{table}[H]
\centering
\begin{tabular}{|>{\centering}p{4cm}|m{12cm}|}
\hline
\rowcolor{gray} \textbf{Lagebeschreibung} & \multicolumn{1}{c}{\textbf{Definition}} \\ \hline
Sehr nah & Dieser Bereich besitzt eine sehr hohe Wahrscheinlichkeit, dass der Nutzer direkt vor einem Beacon steht. Dies gilt für einen Bereich, der in einer Distanz von unter 1 Meter zum Beacon liegt. \\ \hline
Nahbereich & Hier befindet sich der Bereich in einer Sichtlinie zu einem Beacon in einer Distanz von 1 bis 5 Metern. Da es aufgrund von Störungen, wie vorbeilaufender Menschen oder anderer Objekte zur Signalbeeinträchtigung kommen kann, könnte dieser Zustand nicht angezeigt werden, obwohl das Empfangsgerät in diesem Bereich liegt.\\ \hline
Fernbereich & Hier werden zwar die Signale von einem Beacon empfangen, jedoch kann aufgrund der Signalschwäche dem Empfänger kein eindeutiger Bereich zugeordnet werden. Dies impliziert jedoch keine große Entfernung zum Beacon, da wegen der genannten Störungen das Signal möglicherweise verfälscht wurde. Hier müssen weitere Verfahren und Strategien angewendet werden, um den Nutzer und sein Empfangsgerät genauer zu lokalisieren. Beispielsweise kann dem Nutzer empfohlen werden, sein Empfangsgerät höher zu halten oder er sollte ein wenig umherlaufen.\\ \hline
Kein Empfang & Hier wurde ein Beacon nicht erkannt und somit liegt keine physische Nähe zum Beacon vor.\\ \hline
\end{tabular}
\caption{Beispiel der Bereichsdefinition; in Anlehnung an: \cite{GSwiB}}
\label{table:Ranging}
\end{table}
\subsection{Mikro-Lokalisierung}
%\begin{wrapfigure}{r}{8cm}
%\centering
%\begin{tikzpicture}
%\node[anchor=mid,inner sep=0] (Beacon1) at (0,0){\includegraphics[scale=0.1]{Bilder/Beacon}};
%\node[above left] (P1) at (Beacon1) {$P_1$};
%\node[anchor=mid,inner sep=0] (Beacon2) at (6,0){\includegraphics[scale=0.1]{Bilder/Beacon}};
%\node[above right] (P2) at (Beacon2) {$P_2$};
%\node[anchor=mid,inner sep=0] (Beacon3) at (3,5){\includegraphics[scale=0.1]{Bilder/Beacon}};
%\node[above right] (P3) at (Beacon3) {$P_3$};
%\node[anchor=mid,inner sep=0] (MotoG) at (3,2.1){\includegraphics[scale=0.04]{Bilder/MotoG}};
%\node[above right, inner sep=6] (P4) at (MotoG) {$P_{Inter}$};
%\draw [thick, ->] (Beacon1) -- node[above] {$r_1$} ++(MotoG);
%\draw [thick, ->] (Beacon2) -- node[above] {$r_2$} ++(MotoG);
%\draw [thick, ->] (Beacon3) -- node[right] {$r_3$} ++(MotoG);
%\end{tikzpicture}
%\caption{Beispiel einer Trilateration zur Positionsbestimmung}
%\label{fig:Trilat}
%\end{wrapfigure}
Im Gegensatz zur relativen Lokalisierung mittels Lagebeschreibung existiert außerdem die Methode der Mikro-Lokalisierung. Diese berechnet eine genaue Distanz zu einem Sender und letztlich kann mithilfe von zusätzlichen Signalen mehrerer Beacons eine genaue Lokalisierung mit Koordinaten im Raum durchgeführt werden. Für die Bestimmung einer Position aus den empfangenen Beacon-Signalen existieren mehrere Verfahren. Eine gängige Methode ist die Tri- bzw. Multilateration, die aus der Entfernung eines Punktes zwischen drei oder mehreren Orientierungspunkten dessen Position in einem Koordinatensystem bestimmt. In diesem Fall ließen sich die Positionen der Beacons, die Orientierungspunkte und die Entfernungen aus den Signalen der Beacons berechnen. Mit einem Modell der Signalausbreitung wird dabei die Signalstärke, die durch den Weg vom Beacon zum Empfangsgerät geschwächt wurde, direkt in eine Distanz umgerechnet. Im übernächsten Kapitel wird ein solches Modell dafür beschrieben. An dieser Stelle soll noch einmal auf die Vorgehensweise zur Lokalisierung mittels Trilateration eingegangen werden. Abbildung \ref{fig:Trilat} dient zur Veranschaulichung der Lösung. Im ersten Schritt werden dazu die Positionen der Beacons zur Vereinfachung in ein neues Koordinatensystem transformiert. Dabei wird ein Beacon in den Koordinatenursprung verlegt und ein zweiter durch Rotation um die Z-Achse auf die X-Achse gesetzt.
\begin{figure}[H]
\centering
\begin{tikzpicture}
\node[anchor=mid,inner sep=0] (Beacon1) at (0,0){\includegraphics[scale=0.15]{Bilder/Beacon}};
\node[above left, inner sep=6] (P1) at (Beacon1) {$P_1$};
\node[anchor=mid,inner sep=0] (Beacon2) at (6,0){\includegraphics[scale=0.15]{Bilder/Beacon}};
\node[above right, inner sep=6] (P2) at (Beacon2) {$P_2$};
\node[anchor=mid,inner sep=0] (Beacon3) at (3,5){\includegraphics[scale=0.15]{Bilder/Beacon}};
\node[above right, inner sep=6] (P3) at (Beacon3) {$P_3$};
\node[anchor=mid,inner sep=0] (MotoG) at (3,2.1){\includegraphics[scale=0.06]{Bilder/MotoG}};
\node[above right, inner sep=10] (P4) at (MotoG) {$P_{Inter}$};
\draw [thick, ->] (Beacon1) -- node[above] {$r_1$} ++(MotoG);
\draw [thick, ->] (Beacon2) -- node[above] {$r_2$} ++(MotoG);
\draw [thick, ->] (Beacon3) -- node[right] {$r_3$} ++(MotoG);
\end{tikzpicture}
\caption{Beispiel einer Trilateration zur Positionsbestimmung}
\label{fig:Trilat}
\end{figure}
Translation:
\begin{align*}
T &= -P_1\\
P_1' &= P_1 + T = \binom{0}{0}\\
P_2' &= P_2 + T\\
P_3' &= P_3 + T
\end{align*} 
Rotation:
\begin{align*}
\alpha = -\arcsin \left ( \frac{y_2}{\sqrt{x_2^2+y_2^2}} \right )\\
\end{align*} 
\begin{align*}
R_z\left ( \alpha \right ) = \begin{bmatrix}
\cos\left ( \alpha \right ) & -\sin\left ( \alpha \right )\\ 
\sin\left ( \alpha \right ) & \cos\left ( \alpha \right )
\end{bmatrix}\\
\end{align*}
\begin{align*}
P_1'' &= \binom{0}{0}\\
P_2'' &= R_z\left ( \alpha \right ) \cdot P_2'\\
P_3'' &= R_z\left ( \alpha \right ) \cdot P_3'
\end{align*}
Anhand zweier Formeln lassen sich dann die Koordinaten des Empfangsgeräts aus den einzelnen Distanzen berechnen und schließlich ins ursprüngliche Koordinatensystem zurück transformieren \cite{Trilat}. Die Distanzen sind dabei gegeben als $r_1$, $r_2$ und $r_3$ und die Koordinaten entsprechen der Nummerierung der Punkte nach der Transformation.
\begin{align*}
x_{Inter} &= \frac{r_1^2-r_2^2+x_2^2}{2x_2^2}\\
y_{Inter} &= \frac{r_1^2-r_3^2+x_3^2+y_3^2}{2y_3^2}-\frac{x_3}{y_3}\cdot x_{Inter}
\end{align*}
Rücktransformation:
\begin{align*}
P_{Inter} = R_z\left ( -\alpha \right ) \cdot \binom{x_{Inter}}{y_{Inter}} - T
\end{align*}
Ein beträchtlicher Nachteil dieser Methode ist die ausgedehnte Fehleranfälligkeit der Signalübertragung. In der Anwendung in Gebäuden werden die Signale oftmals von Wänden reflektiert oder gedämpft, während die hohe Dichte von Signalquellen weitere Interferenzen verursacht. Um diese Nachteile auszugleichen, wird statt der Trilateration eine Multilateration angewendet. Durch die Nutzung mehrerer Beacons können Messrauschen und Störungen zum Teil ausgeglichen und die Genauigkeit der Lokalisierung somit erhöht werden. Jedoch kann in dieser Arbeit nicht auf dieses Verfahren eingegangen werden, da lediglich drei Beacons zur Verfügung standen. 
\section{Anwendungsbereiche der Indoor-Lokalisierung}
Der Bedarf an einer innerräumlichen Ortung ist normalerweise dort vorhanden, wo es für den Besucher schwer ist, sich zu orientieren. Dies können weitläufige Gebäudekomplexe sein oder verwinkelte und unüberschaubare Gänge. Infolge der Digitalisierung der Gesellschaft (oft als "`Digitale Revolution"' oder in einem anderen Kontext als "`Zweite Moderne"' \cite{DigRev} bezeichnet), entstehen durch den starken Grad der Verbreitung von mobilen "`smarten"' Geräten riesige Netzwerke mit einer schier unendlichen Datenflut. Diese Entwicklung eröffnet folglich eine Möglichkeit zur Kommunikation zwischen Mensch und Gebäude, in der die Beacons als "`Sinne"' des Gebäudes verstanden werden können.  
\subsection{Mögliche Einsatzszenarien}
\begin{wrapfigure}{r}{9cm} 
\centering
\includegraphics[scale=0.3]{Bilder/iBeaconShoe}
\caption{Beacon-Nutzung in Geschäften \cite{Shoe}}
\label{fig:Shoe}
\end{wrapfigure}
Viele Unternehmen nutzen die Möglichkeiten einer vernetzten Welt, um das Verhalten der Kundschaft besser verstehen zu können. Das Beispiel der Warenhauskette zeigt, dass das Wissen für das Unternehmen von Vorteil wäre, wie viele Kunden am Tag eine Filiale besuchen, wie ihr Bewegungsprofil aussieht und was sie am Ende einkaufen. Nützlich wären die Informationen für die Preisgestaltung, die Anordnung der Abteilungen im Warenhaus und die Erkenntnisse daraus könnten auch zu einer besseren "`Just-In-Time"' Lieferkette führen und somit Lagerkapazitäten einsparen. Natürlich wäre dies auch mit herkömmlichen Methoden möglich, indem z.B. Umfragen stattfinden und die Mitarbeiter eines Geschäftes die Kunden genau beobachten würden. Dies wäre aufgrund hoher Personalkosten nicht nur unrentabel, sondern würde den Kunden außerdem ein Gefühl der Überwachung vermitteln. Mit der Beacon-Technologie ist dies kostengünstig und voll automatisiert möglich. Damit potentielle Kunden auch die dafür nötige Software auf ihrem mobilen Endgerät installieren und der Auswertung ihrer Daten zustimmen, können die Kunden mit einer entsprechenden Applikation an exklusiven Gewinnspielen, Rabattaktionen oder Punktesystemen teilnehmen, so wie es heute schon durch einige Unternehmen angeboten wird (z.B. DeutschlandCard, Payback etc.). Die gesellschaftliche Akzeptanz wäre sicherlich gegeben, da schon mit den modernen Stauwarnsystemen ein Vergleichsfall im Outdoor-Bereich existiert. Da die Position eines Fahrzeuges entweder durch das eingebaute Navigationssystem bekannt oder durch die Mobilfunkgeräte der Fahrzeuginsassen ermittelt werden kann, werden schon heute diese Systeme zur Erschaffung von Bewegungsprofilen und damit zur Stauvorhersage genutzt \cite{Stau}. Für ein solches Szenario im Indoor-Bereich würde hauptsächlich die kontextbezogene Lokalisierung in Betracht gezogen werden, da hier die meisten Geschäfte überschaubar sind und schon gute Strukturen zur Orientierung der Kundschaft genutzt werden. Extreme Ausmaße annehmende Einkaufszentren wie die "`Golden Resources Mall"' in Peking mit 557,419 $\text{m}^2$ Ladenfläche, wo eine genauere Lokalisierung zur Navigation sicherlich sinnvoll wäre, bilden hier eher die Ausnahme. \par\bigskip
\begin{wrapfigure}{l}{8.5cm} 
\centering
\includegraphics[scale=0.5]{Bilder/iBeaconAirport}
\caption{Beacon-Nutzung im Flughäfen \cite{Airpo}}
\label{Airpo}
\end{wrapfigure}
Im Gegensatz dazu stellt die Mikro-Lokalisierung ein weiteres mögliches Szenario dar. Sie soll dazu dienen, Menschen oder Objekte genau zu lokalisieren, um unter Verwendung von speziellen Applikationen eine bessere Orientierung zu gewährleisten. Daran interessiert sind meist Branchen, die ihre Kundschaft gerne schnell und ohne Umwege zu dem führen möchten, weswegen sie in die Einrichtung des Unternehmens gekommen sind. Als Beispiele können hier Messehallen- und Flughafenbetreiber genannt werden. In diesen Branchen sind die Gebäude meist sehr unübersichtlich und die Besucher häufig ortsfremd. Der Vorteil, den die Unternehmen aus dem System ziehen können, ist eine bessere Verteilung der Kunden durch eine intelligente Pfadplanung. So können die Platzkapazitäten der Einrichtungen besser genutzt und die Gäste somit schneller abgefertigt werden. Wenn zudem ein Unternehmen in seinen Gebäuden ein ähnliches System wie ein Navigationsgerät für die Straße bieten kann, bedeutet dies auch einen Wettbewerbsvorteil und fördert die Kundenzufriedenheit. \\ \\
Grundsätzlich sind dies hier nur Annahmen für mögliche Szenarien. Da es noch keine praktikablen Anwendungen der Indoor-Lokalisierung gibt, können keine genauen Vorhersagen diesbezüglich getroffen werden. Viele Unternehmen schrecken noch davor zurück, weil kein einheitlicher Standard existiert und nur wenige Menschen mit der Technik vertraut sind. 
\newpage
\subsection{Planungskonzepte für Beacon-Konfigurationen}
\begin{wrapfigure}{r}{6cm}
\centering
\includegraphics[scale=0.23]{Bilder/TrackEstimote}
\caption{Demonstration der Estimote Indoor Localization App \cite{TrEs}}
\label{fig:EstiPlan}
\end{wrapfigure}
Bisherige Ansätze um ein Lokalisierungssystem in einem Raum zu realisieren, beruhen auf fachmännischen Einschätzungen des Beacon-Installateurs. Das bedeutet, dass sich das Konzept lediglich auf Erfahrungswerte beruft. Die Beacons werden dabei an strategische Punkte gesetzt und manuell an jeder Position neu konfiguriert. Wie schon oben beschrieben, werden die Befestigungen der Beacons im Raum vermessen oder grob geschätzt und so zur Positionsbestimmung in eine Lokalisierungs-Applikation ebenfalls wieder manuell eingepflegt. Von der Firma Estimote gibt es seit Neuestem eine Anwendung namens "`Estimote Indoor Localization "', die den Entwickler einer Navigations-App unterstützen soll. Der Entwickler installiert dazu an allen vier Wänden eines Raumes einen Beacon der Firma, startet ausgehend vom Eingang die App und läuft im Uhrzeigersinn den ganzen Raum ab. Dabei werden die Signale der Beacons an den Wänden aufgezeichnet und mithilfe der inertialen Sensoren des Smartphones zusätzlich die Bewegungsinformationen gespeichert. Daraus errechnet die App eine Karte vom Raum und liefert auch gleich eine Positionsbestimmung mithilfe der genannten Sensordaten in Echtzeit. Die Demonstration des Programms ist hierfür in Abbildung \ref{fig:EstiPlan} einmal festgehalten. Der Vorteil dieser App ist es, dass die Karte auch in weiteren Programmen für das Smartphone verwendet werden kann, jedoch ist dafür eine Migration der Daten notwendig. \\ \\
Da für die Experimente in dieser Arbeit nur drei Beacons zur Verfügung standen und diese App erst in der Endphase der Bearbeitungszeit veröffentlicht wurde, wird hier nicht mehr auf dieses Konzept eingegangen. Jedoch lassen die noch engen Begrenzungen der Anwendung (konstante Anzahl von vier Beacons, Einstellung der Parameter wieder manuell nach Erfahrungswert) keinen Handlungsspielraum für die Optimierung der Beacon-Konfigurationen zu. Zudem muss immer ein Entwickler die Räume ablaufen und später die Daten vom Smartphone in die eigene Entwicklung migrieren. Für kleine Räume ist diese Strategie sicherlich hilfreich, jedoch wird sie bei Großprojekten, wie z.B. der Schaffung eines Ortungssystems in einem Flughafen schlicht unbrauchbar.