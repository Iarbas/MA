\chapter{Evaluierung, Zusammenfassung und Ausblick}
Noch einmal erklären, warum jetzt die Überprüfung durchgeführt wird. Und erwähnen, dass mit diesem Test auch die Methode des Fingerprinting umgesetzt werden kann (aber est im Ausblick). 
\section{Aufbau des Parcours}
Ein paar Sätze wie wir des testfeld errichtet haben 

\section{Durchführung}
Vorgehen erklären
\subsection{Planung mittel Lighthouse Keeper}
Bunte Bilder zeigen. Einmal von den Scan-Ergebnissen vom Scitos G5 und dann was der Lighthouse Keeper daraus gemacht hat.
\subsection{Messergebnisse}
Verlinkung vom Video auf Youtube. Aufbereitung der Scitos-Daten. Die empfangene Signalstärke an jedem Messpunkt von jedem Beacon zeigen (3 Bilder)

\section{Auswertung}
Diskussion Simulation zu realen Messwerten.

\section{Zusammenfassung}
Hier noch einmal das Bild von dem Energieverbrauch, der Richterskala, dem Lighthouse Keeper und von der Evaluation zeigen. Betonen, dass es sich hier um Grundlagenforschung handelt und das das Modell noch um Isolinien erweitert werden kann. Es noch viel zu tun gibt bei den Beacons (könnten ja mal den 2,4 Ghz Bereich scannen, ob er frei ist). Das man immer abwägen muss zwischen Genauigkeit und Reichweite der Lokalisierung und Wartungsaufwand. Das die Modelle so nicht eins zu eins übertragbar sind und mein verwendeter Lokalisierungsalgorithmus nur rudimentär ist. Demzufolge kann mit besseren Algorithmen die Qualität der Lokalisierung weiter verbessert werden. Jedoch würde sich dadurch nichts am Optimierungsalgorithmus ändern, sodass er eine gute Grundlage zur automatisierten Konfiguration bildet. Zumindest bietet er gute Positionen für Beacons. 

Ich hoffe, dass diese Arbeit einen Teil dazu beiträgt, die Beacon-Technologie weiter voran zu bringen. 

Einfluss der Intervalllänge und Einfluss der Signalstärke fehlen! Erklärung: wäre sonst zu umfangreich.



Hier steht dann die Zusammenfassung und der Ausblick.


Zusammenfassung ist klar. Der Ausblick könnte sein die Technik in Einkaufhäusern, Parkgaragen, Lagerhallen, etc. einzusetzen. 

Seither gibt es unzählige Firmen, die BLE-Sender herstellen (Liste mit neun größten Herstellern: \url{http://www.nodesagency.com/list-9-biggest-beacon-manufacturers/}, Stand März 2014). Jedoch bietet noch keine Firma eine Komplettlösung an, sondern sie stellen lediglich die Hardware und ein Software Development Kit (SDK) bereit. Sie überlassen es somit den Kunden die Beacons zu installieren, sowie eigene Applikationen für deren Nutzung auf mobilen Geräten zu entwickeln. Dies schreckt möglicherweise potentielle Kunden ab und gleichzeitig signalisiert dies eine große Marktlücke. Wenn jedoch diese gefüllt werden kann, könnte die gesamten Technologie einen neuen Auftrieb erfahren.