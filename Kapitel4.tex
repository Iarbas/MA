\chapter{Ergebnisse und Zusammenfassung}
Hier kommt eher die Simulation und Optimierung hin. Jedenfalls brauche ich noch für den Lighthousekeeper einen manuellen Modus, indm ich die Signalstärke einstellen kann. Die Intervalllänge kann dann nur als intuitive Auslegungsgröße beschreiben werden, da mit mit der Intervalllänge auch eine Messabweichung einher geht. Hier kann angesprochen werden, dass Faktoren wie reflektierende Mauern, Böden, Fensterscheiben, etc. ein schnelleres Intervall benötigen, um eine gewisse Lokalisierungsgenauigkeit zu erreichen. Das würde ich als Erfahrungswert interpretieren.Bei der Optimierung muss hingegen keine Signalstärke oder Intervalllänge angegeben werden, da die Berechnung absolut ist. Die Beacons werden so oder so optimal verteilt. Hier wäre noch anzumerken, dass es noch keinen Modus gibt die optimal verteilten beacons in den Manuell-Modus zu übernehmen und dann mit der Signalstärke zu experimentieren. Noch muss man sich die ungefähren Positionen merken.






Hier steht dann die Zusammenfassung und der Ausblick.


Zusammenfassung ist klar. Der Ausblick könnte sein die Technik in Einkaufhäusern, Parkgaragen, Lagerhallen, etc. einzusetzen. 

Seither gibt es unzählige Firmen, die BLE-Sender herstellen (Liste mit neun größten Herstellern: \url{http://www.nodesagency.com/list-9-biggest-beacon-manufacturers/}, Stand März 2014). Jedoch bietet noch keine Firma eine Komplettlösung an, sondern sie stellen lediglich die Hardware und ein Software Development Kit (SDK) bereit. Sie überlassen es somit den Kunden die Beacons zu installieren, sowie eigene Applikationen für deren Nutzung auf mobilen Geräten zu entwickeln. Dies schreckt möglicherweise potentielle Kunden ab und gleichzeitig signalisiert dies eine große Marktlücke. Wenn jedoch diese gefüllt werden kann, könnte die gesamten Technologie einen neuen Auftrieb erfahren.

