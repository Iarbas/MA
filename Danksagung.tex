\chapter*{Danksagung}
Die vorliegende Arbeit entstand am Lehrstuhl für Eingebettete Smarte Systeme (ESS) der Otto-von-Guericke Universität Magdeburg. \par\smallskip
Einen besonderen Dank möchte ich an dieser Stelle meinem Betreuer, Jun.-Prof. Dr.-Ing. Sebastian Zug, für seine Unterstützung, Förderung und dem mir entgegengebrachten Vertrauen im Laufe meiner Arbeit aussprechen. Er stellte mir ein sehr gutes Arbeitsumfeld zur Verfügung, in dem ich ungehindert forschen und mich auf mein Thema konzentrieren konnte. Vielen Dank auch an meinen Zweitprüfer Prof. Dr.-Ing. Abbas Omar und seinem Doktoranten Abdo Nasser Ali Gaber für den Einblick in ihre Forschungen und den damit verbunden Inspirationen für mein Thema. \par\smallskip
Ich möchte auch das Stipendium mit finanzieller Unterstützung der Stiftung Industrieforschung nicht vergessen, das mir mehr Unabhängigkeit und Freiheit in meiner Arbeit ermöglichte. Vielen Dank dafür. \par\smallskip
Viele Menschen haben einen signifikanten Beitrag zu dieser Arbeit geleistet. Als erstes möchte ich Christoph Steup nennen, der sich mit mir den Arbeitsplatz teilte und in anregenden Diskussionen, ob professionell oder persönlich, mir neuen Antrieb schenkte. Speziellen Dank auch an Dirk Steindorf, der seine Expertise mit dem Scitos G5 Roboter mit mir teilte und mich in der Evaluationsphase tatkräftig unterstützte. \par\smallskip
Meinen langjährigen Kommilitonen und Freunden möchte ich für die Unterstützung im Studium danken. Allen voran Andreas Himmel und Hannes Heinemann, deren Zukunft ihnen hoffentlich nur Erfolg verspricht. Mit beiden habe ich viele Höhen und Tiefen erlebt und gemeinsam haben wir diese gemeistert. Ich möchte auch Julian Scholle nicht vergessen, der zwischenzeitlich neue Wege ging und mich in die Welt der Informatik mitnahm. \par\smallskip
Zuletzt möchte ich mich herzlich bei meinen Eltern und meiner Partnerin Carolin Richter für ihre Unterstützung, Geduld und Verständnis bedanken.\par\medskip
\begin{flushright}
André Pieper\\
Magdeburg, Mai 2015
\end{flushright}  